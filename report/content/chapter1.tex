\section{问题背景与重述}

\subsection{问题背景}
对于以装配测试为主的生产型企业,员工的技能水平是影响生产的一项重要因素。但是,受到复杂因素的影响,员工的流动往往较为频繁,而新员工入职后需要经过专业培训才能进入产线工作,因此在新老员工交替阶段,产线的产能和产品质量往往会产生较大的波动,进而可能会导致严重的后果。该问题可以通过恰当的员工管理和产线分配策略解决。

\subsection{问题提出}
现有某生产企业40天内的生产订单数据、20名员工具体的技能水平数据和新员工的培训要求,要求对目前对生产员工培训与产线分配的策略提出优化方案,使得产能在员工流动时保持基本的稳定,新员工可以无缝地加入后续生产队列。问题的设置是逐渐加强限制条件,并服务于同一主题的:

\begin{enumerate}[label=(\arabic*)]
    \item \textbf{问题一}:在企业存在足够多员工、员工在所有产线上的技能水平均为 E、不考虑培训与升级的情况下,设计最优分配方案,使得所有订单的超时分钟数之和最小,并给出超时分钟数之和。
    \item \textbf{问题二}:在员工数量有限、员工技能水平存在差异、员工不能进行培训与升级、员工无增减变化的情况下,重复问题一的任务。
    \item \textbf{问题三}:在员工数量有限、员工技能水平存在差异、员工可以培训升级、员工无增减变化的情况下,重复问题一二的任务,使得完成订单的同时尽可能提升员工的技能水平(必要时可适当牺牲一些超时分钟数),并给出最优的培训方案。
    \item \textbf{问题四}:在问题三的生产计划分配的基础上,假设在第2250分钟,员工PE001到PE010提出离职申请,并在第4500分钟离职,同时在第6750分钟,10名新员工将入职,其所有技能水平均为N。给出相对较优的生产计划调整方案,使得所有订单的超时分钟数之和最小,说明方案的合理性,并给出超时分钟数之和。
\end{enumerate}
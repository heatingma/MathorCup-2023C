\section{问题分析}

\subsection{问题一分析}
本问题情景类似于操作系统的进程调度。我们采用\textbf{贪心}的思想,考虑每完成一个订单后,以五种不同的标准选择下一个生产的订单:完成时间最短优先,剩余时间最短优先,完工截止时间最早、完成时间最短优先,完工截止时间最早、剩余时间最短优先,以及超时订单数最小、完成时间最短优先。比较所有算法结果的优劣,并从理论上给出合理的解释。

\subsection{问题二分析}
由于每条产线在做每个订单、每件产品在生产的时候员工不能替换,因此考虑将员工分配到不同的产线,再采用问题一的最优算法meonf计算得到总超时时间。经过问题一的分析,发现某些产线生产压力较大,超时现象难以避免,而另外一些产线生产压力较小,几乎不存在超时的问题。因此,为了最小化总超时时间,分配员工时应尽量使得生产压力越大的产线分到的员工技能水平越高。以此为目标建立\textbf{线性整数规划模型},求解最优员工分配矩阵。


\subsection{问题三分析}
本题在前述问题的基础上增加一条“员工可以培训升级”的条件。根据题意,我们需要同时考虑尽可能提升员工技能水平和最小化总超时时间两个目标,因此在问题一和问题二的基础上建立\textbf{双目标优化模型},并将问题二的分配矩阵作为初始方案,利用\textbf{贪心}的思想,设计了算法进行求解。

\subsection{问题四分析}
由于存在员工流动的情况,考虑根据关键时间点将时间分段,不同时间段内根据员工生产情况对问题三的双目标函数参数进行调整,使得特定时间段对关键员工技能水平提升的考虑权重加大,保证生产进度的稳定性。
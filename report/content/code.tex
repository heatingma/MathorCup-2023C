\newpage
\section{代码}
这里只列举出部分代码,完整代码仓库在 \href{https://github.com/heatingma/MathorCup-2023C}{https://github.com/heatingma/MathorCup-2023C}。

\subsection{数据处理代码}

\begin{lstlisting}[caption={数据处理}, label={lst:python}]
import pandas as pd
import numpy as np
import pdb

def divide_data():
    """
    divide data to 12 lines
    """
    df = pd.read_excel("data/1 - 生产订单.xlsx", usecols=np.r_[0:5])
    data = np.array(df)
    lines = dict()
    for i in range(12):
        lines[i] = list()        
    for order in data:
        lines[int(order[4][-2:]) - 1].append(order)
    for i in range(12):
        np.save("processed_data/problem_1/line_{}".format(i+1), lines[i])
        lines[i] = pd.DataFrame(data=lines[i], columns=df.columns)
        lines[i].to_excel("processed_data/problem_1/line_{}.xlsx".format(i+1))


def get_result_from_problem_1():
    data = pd.read_excel("result/problem_1/results-overtime.xlsx")
    data = np.array(data)[0][2:-1].astype(int)
    np.save("processed_data/problem_2/overtime.npy", data)

    data = pd.read_excel("result/problem_1/results-sumtime.xlsx")
    data = np.array(data)[0][2:].astype(int)
    np.save("processed_data/problem_2/sumtime.npy", data)    
    
    data = pd.read_excel("result/problem_1/results-add-sumtime.xlsx")
    data = np.array(data)[0][2:].astype(int)
    np.save("processed_data/problem_2/overtime-sumtime.npy", data)
    

def get_workers():
    data = pd.read_excel("data/2 - 生产员工技能表.xlsx", usecols=np.r_[1:13])
    data = np.array(data)[1:]
    swap = {'E': 1.0, 'O': 0.8, 'N': 0.0}
    for i in range(data.shape[0]):
        for j in range(data.shape[1]):
            data[i][j] = swap[data[i][j]]
    data = data.astype(float)
    np.save("processed_data/problem_2/workers.npy", data)


if __name__ == "__main__":
    divide_data()
    get_result_from_problem_1()
    get_workers()

\end{lstlisting}

\subsection{问题一算法代码}

\begin{lstlisting}[caption={meonf}, label={lst:python}]
def meonf(orders:ORDERS, work_num):
    """
    Min Excpeted Overtime Numbers First
    """
    num_orders = len(orders.orders_dict)
    finished = 0
    time = 0
    
    deadline_orders = dict()
    added = dict()
    for order in orders.orders_dict.values():
        if order.deadline not in deadline_orders.keys():
            added[order.deadline] = False
            deadline_orders[order.deadline] = list()
        deadline_orders[order.deadline].append(order)
        
    deadlines = sorted(deadline_orders)    
    nums_ddl = len(deadlines)
    cur_deadline = deadlines[0]
    if nums_ddl > 1:
        next_ddl_id = 1
        next_deadline = deadlines[next_ddl_id]
    orders_cache = deadline_orders[cur_deadline]
    
    orders_cache: list()
    while(finished != num_orders):
        # action for next deadline coming
        flag = True
        for order in orders_cache:
            if order.finished == False:
                flag = False 
        if time > cur_deadline or flag:
            if next_deadline != -1 and added[next_deadline] == False:
                added[next_deadline] = True
                cur_deadline = next_deadline
                next_ddl_id += 1
                next_deadline = deadlines[next_ddl_id] if next_ddl_id < nums_ddl else -1
                for order in deadline_orders[cur_deadline]:
                    orders_cache.append(order)

        # find the order to achieve the minimal expected overtime numbers
        order_id = None
        min_overtime_nums = None
        min_left_time = None
        
        for order in orders_cache:
            order: ORDER
            if order.finished:
                continue
            vir_time = time + order.sum_time
            nums = cal_overtime_nums(orders_cache, vir_time)
            
            if min_overtime_nums is None or nums < min_overtime_nums:
                min_overtime_nums = nums
                order_id = order.id
                min_left_time = order.left_time
            elif nums == min_overtime_nums and min_left_time > order.left_time:
                order_id = order.id
                min_left_time = order.left_time                
                 
        order = orders.orders_dict[order_id]
        order: ORDER
        
        # begin work
        if order.begin_time is None:
            order.begin_order(time)
        time, num = order.begin_work(work_num, time)
        finished += num
        
    return orders
    
def cal_overtime_nums(orders_cache, vir_time):
    nums = 0
    for order in orders_cache:
        order: ORDER
        if order.finished:
            continue
        if order.deadline < vir_time:
            nums += 1
    return nums

\end{lstlisting}


\begin{lstlisting}[caption={derf}, label={lst:python}]
def derf(orders: ORDERS, work_num):
    """
    Dealine & Earliest Remaining Time First
    """
    num_orders = len(orders.orders_dict)
    finished = 0
    time = 0
    
    while(finished != num_orders):
        # find the minimal remaining time's order
        min_deadline = None
        min_rm_time = None
        order_id = None
        for id, order in orders.orders_dict.items():
            order: ORDER
            if order.finished:
                continue
            if min_deadline is None or min_deadline > order.deadline:
                min_deadline = order.deadline
                min_rm_time = order.sum_time
                order_id = id
            elif min_deadline == order.deadline and min_rm_time > order.rm_time:
                min_rm_time = order.sum_time
                order_id = id
                    
        order = orders.orders_dict[order_id]
        order: ORDER
        # begin work
        if order.begin_time is None:
            order.begin_order(time)
        time, num = order.begin_work(work_num, time)
        finished += num
    return orders
\end{lstlisting}


\begin{lstlisting}[caption={erf}, label={lst:python}]
def erf(orders: ORDERS, work_num):
    """
    Earliest Remaining Time First
    """
    num_orders = len(orders.orders_dict)
    finished = 0
    time = 0
    
    while(finished != num_orders):
        # find the minimal remaining time's order
        min_rm_time = None
        order_id = None
        for id, order in orders.orders_dict.items():
            order: ORDER
            if order.finished:
                continue
            if min_rm_time is None or min_rm_time > order.rm_time:
                min_rm_time = order.rm_time
                order_id = id
        order = orders.orders_dict[order_id]
        order: ORDER
        # begin work
        if order.begin_time is None:
            order.begin_order(time)
        time, num = order.begin_work(work_num, time)
        finished += num
    return orders
\end{lstlisting}

\begin{lstlisting}[caption={msf}, label={lst:python}]
def msf(orders: ORDERS):
    """
    Min Sum Time First
    """
    num_orders = len(orders.orders_dict)
    finished = 0
    time = 0
    
    while(finished != num_orders):
        # find the minimal remaining time's order
        min_sum_time = None
        order_id = None
        for id, order in orders.orders_dict.items():
            order: ORDER
            if order.finished:
                continue
            if min_sum_time is None or min_sum_time > order.sum_time:
                min_sum_time = order.sum_time
                order_id = id
        order = orders.orders_dict[order_id]
        order: ORDER
        # begin work
        time = order.begin_order(time)
        finished += 1
    return orders
\end{lstlisting}

\begin{lstlisting}[caption={dmsf}, label={lst:python}]
def dmsf(orders: ORDERS):
    """
    Dealine & Min Sum Time First
    """
    num_orders = len(orders.orders_dict)
    finished = 0
    time = 0
    
    while(finished != num_orders):
        min_deadline = None
        min_sum_time = None
        order_id = None
        for id, order in orders.orders_dict.items():
            order: ORDER
            if order.finished:
                continue
            if min_deadline is None or min_deadline > order.deadline:
                min_deadline = order.deadline
                min_sum_time = order.sum_time
                order_id = id
            elif min_deadline == order.deadline and min_sum_time > order.sum_time:
                min_sum_time = order.sum_time
                order_id = id
                    
        order = orders.orders_dict[order_id]
        order: ORDER
        # begin work
        time = order.begin_order(time)
        finished += 1
    return orders
\end{lstlisting}

\subsection{问题二算法代码}
\begin{lstlisting}[caption={线性整数规划}, label={lst:python}]
import pulp

def lip(work_num, workers, times):
    """
    Linear Integer Programming
    """
    ots = times / np.sum(times)
    # create problem
    problem = pulp.LpProblem("Worker Allocation", pulp.LpMaximize)
    # target
    x = [[pulp.LpVariable(f"x_{i}_{j}", lowBound=0, upBound=1, cat=pulp.LpInteger) for j in range(12)] for i in range(20)]
    # object
    object = pulp.lpSum(ots[j] * x[i][j] * workers[i][j]  for i in range(20) for j in range(12))
    problem += object
    # constraint
    for i in range(20):
        problem += pulp.lpSum(x[i][j] for j in range(12)) <= 1
    for j in range(12):
        problem += pulp.lpSum(x[i][j] for i in range(20)) <=  work_num[j] 
    # get result
    problem.solve()
    result = np.zeros(shape=(20, 12))
    for i in range(20):
        for j in range(12):
            result[i][j] = pulp.value(x[i][j]) 
    return result 
\end{lstlisting}

\subsection{问题解决代码}
\begin{lstlisting}[caption={问题解决}, label={lst:python}]
import numpy as np
import pandas as pd
from utils import get_orders, get_lines, get_workers, constant
from models import dmsf, msf, derf, erf, meonf, lip


work_num = [1, 2, 1, 3, 1, 1, 2, 1, 1, 3, 1, 1]

def problem_1():
    models = {'meonf': meonf, 'dmsf': dmsf,'derf': derf, 'msf': msf, 'erf': erf}
    # solve 
    for name, model in models.items():
        for i in range(12):
            data = np.load("processed_data/problem_1/line_{}.npy".format(i+1), allow_pickle=True)
            orders = get_orders(data)
            orders = model(orders)
            result = orders.print()
            df = pd.DataFrame(result, columns=['生产订单', '开始时间', '完工时间', '截止时间', '超时', '总耗时'])
            df.to_excel("result/problem_1/{}/result_{}.xlsx".format(name, i+1))
    
    # statistic
    results = list()
    results_2 = list()
    results_3 = list()
    for name in models.keys():
        part_result = [name]
        part_result_2 = [name]
        part_result_3 = [name]
        for i in range(12):
            data = np.array(pd.read_excel("result/problem_1/{}/result_{}.xlsx".format(name, i+1)))
            part_result.append(np.sum(data[:, 5]))
            part_result_2.append(np.sum(data[:, 6]))
            part_result_3.append(np.sum(data[:, 5]) + np.sum(data[:, 6]))
        part_result.append(np.sum(part_result[1:]))
        results.append(part_result)
        results_2.append(part_result_2)
        results_3.append(part_result_3)
        
    rdf = pd.DataFrame(results, columns=['策略', 'Line01', 'Line02', \
        'Line03', 'Line04', 'Line05', 'Line06', 'Line07', 'Line08', \
        'Line09', 'Line10', 'Line11', 'Line12', 'SUM'])
    rdf.to_excel("result/problem_1/results-overtime.xlsx")

    rdf = pd.DataFrame(results_2, columns=['策略', 'Line01', 'Line02', \
        'Line03', 'Line04', 'Line05', 'Line06', 'Line07', 'Line08', \
        'Line09', 'Line10', 'Line11', 'Line12'])
    rdf.to_excel("result/problem_1/results-sumtime.xlsx")
       
    rdf = pd.DataFrame(results_3, columns=['策略', 'Line01', 'Line02', \
        'Line03', 'Line04', 'Line05', 'Line06', 'Line07', 'Line08', \
        'Line09', 'Line10', 'Line11', 'Line12'])
    rdf.to_excel("result/problem_1/results-overtime-sumtime.xlsx")


def problem_2():
    workers = np.load("processed_data/problem_2/workers.npy", allow_pickle=True)
    times_list = ['sumtime', 'overtime-sumtime']
    
    lines = list() 
    for time in times_list:
        times =  np.load("processed_data/problem_2/{}.npy".format(time), allow_pickle=True)
        result = lip(work_num, workers, times)
        swap = {1.: 'E', 0.8: 'O', 0.: 'N'}   
        for j in range(12):
            cur_line = [time, 'line{}'.format(j+1)]
            for i in range(20):
                if result[i][j]:
                    cur_line.append(i+1)
                    cur_line.append(swap[workers[i][j]])
            lines.append(cur_line)
            
    rdf = pd.DataFrame(lines, columns=['采用的时间', '产线', '工人1', '技能1', \
        '工人2', '技能2', '工人3', '技能3'])
    rdf.to_excel("result/problem_2/assign_results.xlsx")



def problem_3():
    workers_data = np.load("processed_data/problem_2/workers.npy", allow_pickle=True)
    workers = get_workers(workers_data)
    lines = get_lines(workers)
    time = 0
    assign = pd.read_excel("result/problem_2/assign_results.xlsx", usecols=[3, 5, 7])
    assign = np.array(assign).astype(int)
    while lines.finished == False:  
        if time == 0:
            lines.update_state(time)
            lines.time_zero(assign)
            time += 1
        else:          
            lines.update_state(time)
            lines.next_orders(time)
            time += 1 

    rdf = pd.DataFrame(lines.get_msg(), columns=['产线', '订单ID', '开始时间', '结束时间', '截止时间', \
        '超时时间', '总耗时', '工人1', '工人2', '工人3'])
    rdf.to_excel("result/problem_3/results_orders.xlsx")
    
    worker_update = np.array(lines.workers.get_msg())
    update = worker_update[:, 13:25] - worker_update[:, 1:13]
    worker_update = np.concatenate([worker_update, update], axis=1)
    
    rdf = pd.DataFrame(worker_update, columns=['工人ID', 'Line01', 'Line02', \
        'Line03', 'Line04', 'Line05', 'Line06', 'Line07', 'Line08', \
        'Line09', 'Line10', 'Line11', 'Line12', 'Line01', 'Line02', \
        'Line03', 'Line04', 'Line05', 'Line06', 'Line07', 'Line08', \
        'Line09', 'Line10', 'Line11', 'Line12', 'Line01', 'Line02', \
        'Line03', 'Line04', 'Line05', 'Line06', 'Line07', 'Line08', \
        'Line09', 'Line10', 'Line11', 'Line12'])
    rdf.to_excel("result/problem_3/worker_update.xlsx")
    
    worker_update = np.array(lines.workers.get_msg_2())
    rdf = pd.DataFrame(worker_update, columns=['工人ID', 'Line01', 'Line02', \
        'Line03', 'Line04', 'Line05', 'Line06', 'Line07', 'Line08', \
        'Line09', 'Line10', 'Line11', 'Line12', 'Line01', 'Line02', \
        'Line03', 'Line04', 'Line05', 'Line06', 'Line07', 'Line08', \
        'Line09', 'Line10', 'Line11', 'Line12'])
    
    rdf.to_excel("result/problem_3/worker_update_2.xlsx")


def problem_4():
    workers_data = np.load("processed_data/problem_2/workers.npy", allow_pickle=True)
    workers = get_workers(workers_data)
    lines = get_lines(workers)
    time = 0
    assign = pd.read_excel("result/problem_2/assign_results.xlsx", usecols=[3, 5, 7])
    assign = np.array(assign).astype(int)
    while lines.finished == False:  
        if time == 0:
            lines.update_state(time)
            lines.time_zero(assign)
            time += 1
            continue
        if time == 2250:
            constant.important_score()
        elif time == 4500:
            constant.lower_score()
        elif time == 6750:
            lines.workers.del_worker([i+1 for i in range(10)])
            lines.workers.get_new_worker(10)
            constant.important_score()
        elif time == 13500:
            constant.lower_score()
        lines.update_state(time)
        lines.next_orders(time)
        time += 1 

    rdf = pd.DataFrame(lines.get_msg(), columns=['产线', '订单ID', '开始时间', '结束时间', '截止时间', \
        '超时时间', '总耗时', '工人1', '工人2', '工人3'])
    rdf.to_excel("result/problem_4/results_orders.xlsx")
    
    worker_update = np.array(lines.workers.get_msg())
    update = worker_update[:, 13:25] - worker_update[:, 1:13]
    worker_update = np.concatenate([worker_update, update], axis=1)
    
    rdf = pd.DataFrame(worker_update, columns=['工人ID', 'Line01', 'Line02', \
        'Line03', 'Line04', 'Line05', 'Line06', 'Line07', 'Line08', \
        'Line09', 'Line10', 'Line11', 'Line12', 'Line01', 'Line02', \
        'Line03', 'Line04', 'Line05', 'Line06', 'Line07', 'Line08', \
        'Line09', 'Line10', 'Line11', 'Line12', 'Line01', 'Line02', \
        'Line03', 'Line04', 'Line05', 'Line06', 'Line07', 'Line08', \
        'Line09', 'Line10', 'Line11', 'Line12'])
    rdf.to_excel("result/problem_4/worker_update.xlsx")
    
    worker_update = np.array(lines.workers.get_msg_2())
    rdf = pd.DataFrame(worker_update, columns=['工人ID', 'Line01', 'Line02', \
        'Line03', 'Line04', 'Line05', 'Line06', 'Line07', 'Line08', \
        'Line09', 'Line10', 'Line11', 'Line12', 'Line01', 'Line02', \
        'Line03', 'Line04', 'Line05', 'Line06', 'Line07', 'Line08', \
        'Line09', 'Line10', 'Line11', 'Line12'])
    
    rdf.to_excel("result/problem_4/worker_update_2.xlsx")

if __name__ == "__main__":
    # problem_1()
    # problem_2()
    # problem_3()
    problem_4()
\end{lstlisting}
\section{问题四的模型建立与求解}
\subsection{问题分析}

对于员工离职和新入职的情况,需要考虑员工水平的提升来保证E等级和O等级的离职员工“后继有人”,而第三问采用的双目标多层优化模型主要考虑了这一点,因此本问题可以继续沿用这个模型。
\subsection{模型介绍}

为了最小化总超时时间,本题依据关键的时间点将时间分段,在不同的时间段根据员工情况和生产情况调整模型的参数,以采取相应的措施:
\begin{itemize}[left=1em]
    \item 0-2250分钟:按问题三的生产计划分配正常进行。
    \item 2250-4500分钟:提高模型中“员工技能水平提升”部分的权重,并取消员工PE001 到 PE010的培训,使员工PE011到 PE020的技能水平得到更快速的提升,以面对即将到来的离职潮,保证整体生产进度的鲁棒性\cite{robust}。
    \item 4500-6750分钟:降低型中“员工技能水平提升”部分的权重,使得剩余的10名员工以最高的效率处理订单。
    \item 6750-13500分钟:提高模型中“员工技能水平提升”部分的权重,使新入职的10名员工尽快升级,加入到订单的生产中。
    \item 13500分钟以后:降低型中“员工技能水平提升”部分的权重,回到初始值,继续按问题三的模型正常进行。
\end{itemize}

\subsection{模型求解结果}
模型最终得到的超时分钟数之和为 4422553.583。分配方案部分见\cref{tab:problem4},完整见附件。

\subsection{合理性分析}
分析第三问和第四问员工分配结果的差异性,可以看出第四问的分配方式使得在2250分钟(即确定要调离员工)之后着重培养11-20号员工。而在员工新入职得6750-13500分钟时间内,也有更多的员工参与到培训中去,这虽然使短时间内得生产效率降低了,但是对于整体的超时时间起到了优化作用。


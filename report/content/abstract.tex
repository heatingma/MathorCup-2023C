\begin{abstract}
	\normalsize
	本文参考进程调度、贪心算法、规划模型等多种思想,建立了一个基于最短完成时间优先(meonf),结合最大化员工技能提升和最小化订单最小超时总和的\textbf{双目标多层贪心优化模型},为生产企业解决员工流动导致的产能和质量问题。
	
	\textbf{针对问题一},本文借鉴操作系统\textbf{进程调度}算法的思想,结合\textbf{贪心算法},考虑每完成一个订单后,以五种不同的标准选择下一个订单进行处理:最短完成时间优先(msf),最短剩余时间优先(erf),最早完工截止时间、最短完成时间优先(dmsf),最早完工截止时间、最短剩余时间优先(derf),以及最小超时订单数、最小超时订单数及最短完成时间优先(meonf)。然后对不同算法的结果进行比较与分析,发现\textbf{meonf算法}得到的结果最优,总超时时间为2125455分钟。最后对算法给出理论解释,说明其合理性。
	
	\textbf{针对问题二},首先,我们发现不同产线的生产压力明显不同,据此将12条产线分为3类,分别为\textbf{繁忙型产线}、\textbf{空闲型产线}和\textbf{周期型产线}。其次,遵循“生产压力越大的产线分配技能水平越高的员工”的原则,将员工分配到不同的产线,以此为目标建立\textbf{线性整数规划模型}求解最优员工分配矩阵。然后对于周期型产线进行\textbf{二次规划}最后,将员工分配到产线,利用问题一的算法计算,得到最短总超时时间为2125455分钟。
	
	\textbf{针对问题三},基于问题一、二建立一个兼顾员工技能提升和订单最小超时总和的\textbf{双目标优化模型},采用问题二的分配矩阵作为初始方案,随每个时间步长采用\textbf{meonf算法}确定订单选择,并参考空闲工人池,结合\textbf{双目标函数最优化}通过多层贪心不断迭代更新该分配矩阵,最终确定生产员工培训与产线分配模型。
	
	\textbf{针对问题四},考虑到不同时间段内存在员工离职,新员工加入的情况,在问题三的模型上针对不同时间段进行\textbf{双目标函数参数的调整},使得特定时间段对关键员工技能水平提升的考虑权重加大,能够适应员工的离职和新员工的培训目标。然后通过分配和培训结果分析该方案的合理性。
	
	\textbf{关键字}:\textbf{贪心算法} \quad \textbf{参数调整} \quad  \textbf{线性整数规划} \quad \textbf{双目标优化}\quad
\end{abstract}


